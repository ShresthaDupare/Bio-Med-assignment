\documentclass[11pt]{article}
\usepackage[utf8]{inputenc}

\title{Assignment 3 Biomed}
\author{A056 Shrestha Dupare}
\date{February 2022}

\begin{document}

\maketitle

\section{Introduction}

Cancer and diabetes may join polio as diseases that have been eradicated in twenty years. Prevention and early diagnosis, we believe, will be crucial in the future of health. In rare circumstances, the development of sickness could be postponed or completely avoided. Because of advanced testing and instruments, most diagnosis (and treatment) may be done at home.


By 2040, we expect the consumer will be at the center of the health model. Interoperable, always-on data will promote closer collaboration among industry stakeholders, and new combinations of services will be offered by incumbents and new entrants (disruptors). Interventions and treatments are likely to be more precise, less complex, less invasive, and cheaper.


\section{What is the future of health}

The future of health that we envision is only around 20 years off, yet health in 2040 will be a world apart from what we have now. We may reasonably expect digital transformation—enabled by fundamentally interoperable data, artificial intelligence (AI), and open, secure platforms—to drive much of this change, based on coming technology. Unlike today, we believe care will be organised around the consumer, rather than around the institutions that drive our existing health care system.


Streams of health data, together with data from a variety of other relevant sources, will merge by 2040 (and possibly much before) to generate a complex and highly tailored picture of each consumer's well-being. Wearable devices that track our steps, sleep habits, and even heart rate have become a part of our daily lives in ways we couldn't have imagined only a few years ago. Many medtech companies are already incorporating always-on biosensors and software into data-generating, data-gathering, and data-sharing devices. Consumers—armed with this highly specific personal information about their own health—will undoubtedly demand that their health information be portable.


\section{Why does the future of health matter?}

Nothing is more vital to our well-being than our health. To varying degrees, we all interact with the health-care system, and we will continue to do so throughout our lives. Individuals, families, and companies, as well as municipal, state, and federal budgets, are all affected by the expense of health care.


We predict that more health spend will be devoted to sustaining well-being and preventing illness by 2040, while less will be tied to assessing conditions and treating illness. Greater emphasis on well-being and identifying health risks earlier will result in fewer and less severe diseases, which will reduce health care spending, allowing the reinvestment of this well-being dividend to expand the benefits to the broad population.


We don't expect sickness to be completely eradicated by 2040, but the application of actionable health insights—driven by interoperable data and intelligent AI—could aid in early detection, proactive intervention, and better understanding of disease development.


\section{Impacts of the future of health}

Companies centred on technology, such as Google, Amazon, and Apple, are beginning to disrupt and transform the market. Legacy stakeholders should think about whether they should disrupt themselves or isolate and protect their services in order to keep part of their market share. Some incumbent businesses may succumb to competition coming from outside the established industry boundaries, while others may be able to help usher in the future of health.


Data from numerous sources will be needed in the future of health to improve research, assist innovators in developing analytic tools, and create the insights needed for personalised, always-on decision-making. As they provide the infrastructure to engage customers, simplify data access and analysis, and link stakeholders throughout the sector, organisations focused on data and platforms can capture an increasingly significant piece of the profit pool. These archetypes will form the foundation of tomorrow's health-care ecosystem.

\end{document}
