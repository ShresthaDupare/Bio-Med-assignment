\documentclass[11pt]{article}
\usepackage[utf8]{inputenc}

\title{Emerging Technologies in Healthcare}
\author{A056 Shrestha Dupare}
\date{23 February 2022}

\begin{document}

\maketitle

\section{NFTs in Healthcare}

2020 saw the ascent of non-fungible tokens, or NFTs. Not at all like cryptographic forms of money, which are utilized as a type of cash or advanced resource, NFTs are expected to be what could be compared to a testament of genuineness. They act to allow to possess a piece of a computerized portrayal of an unmistakable resource . Your wellbeing information, drug medications and components of the human structure, for example, blood can be addressed by NFTs.


Blood donation organisations as of now utilize NFTs for blood donations. Blood givers are set apart with a particular token that can then be followed throughout the system. The donation can then be followed from transport to the clinic, into a blood donation center, and into its inevitable beneficiary.


Then again, the utilization case for production of drugs utilizing NFTs could check a specific clump of medications, making it both simpler to track and stamp it off as real.


\section{Generative AI}

Presented in 2014 by eminent AI scientist and afterward University of Montreal Ph.D. understudy Ian Goodfellow, generative AI models are frameworks that produce "portrayals of this present reality," like sounds and pictures. The reasoning is that the more a machine can really find out with regards to this present reality, the better it can comprehend and decipher what it sees.


The potential of generative AI in healthcare in terms of image generation and analysis is enormous. Rather than people creating neural networks for computers to utilise as a foundation, generative AI allows machines to augment those models by creating their own. Earlier detection of potential malignancy to more successful treatment strategies depends on the ability to develop new models — genuine content that can be gradually taught from.


\section{Bioprinting: Creating New Organs}

Bioprinting, like 3D printing, is an additive manufacturing technology that employs a digital file as a blueprint to print an object layer by layer. Bioprinters, unlike 3D printers, use cells and biomaterials to create organ-like structures that allow living cells to reproduce. Bioprinting is a relatively young technique with enormous potential in fields such as health and aesthetics.


As well as keeping organs alive outside of the body, different choices ought to likewise be investigated. Despite the fact that it might seem like sci-fi, 3D printed organs are an undeniable, albeit creating, innovation that has effectively advanced into clinical testing. Ears, corneas, bones, and skin are generally organs in clinical testing for 3D bioprinting.The process isn't excessively not the same as conventional 3D printing. Initial an advanced model of the tissue should be made. Cautious consideration should be paid to the resolution and matrix structure, as the materials utilized in the printing system are straightforwardly living cells called bioink. They then, at that point, need to test the organ's usefulness.


\section{Smart Pills}

One of the most significant applications for IoT innovation in medical care is the idea of a smart pill, which changes The Internet of Things into The Internet of Bodies. Smart pills are eatable hardware that fill in as drugs, yet can give care suppliers significant data regarding patients. The first smart pill supported by the FDA was delivered in 2017.


\section{Health/Vaccine Passport}

An vaccine passport is confirmation that you've tried negative for or been safeguarded against specific diseases. It tends to be computerized, similar to a mobile application, or physical, for example, a little paper card. You can convey it with you and show it whenever required, as before you go into the workplace, board a plane, or visit a café, cinema, or rec center.


\end{document}
